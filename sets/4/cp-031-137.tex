For the circuit in Fig. 1-31\emph{b}, find: (\emph{a}) $i_1$; (\emph{b}) $i_2$; (\emph{c}) $i_3$; (\emph{d}) $i_4$.
\begin{align*}
	0.2V_1 &= \frac{V_1}{25}+\frac{V_1}{10}+\frac{V_1}{100}+2.5 \\
	-2.5 &= 0.04V_1-0.2V_1+0.1V_1+0.01V_1 \\
	V_1&=50\qw{\V}	
\end{align*}
\begin{align*}
	i_{100} &= \frac{V_1}{100\qw{\ohm}} = 0.5\qw{\A} \\
	i_{25} &= \frac{V_1}{25\qw{\ohm}} \ = 2\qw{\A} \\
	i_{10} &= \frac{V_1}{10\qw{\ohm}} \ = 5\qw{\A} \\
\end{align*}
\begin{enumerate}[leftmargin=2cm,labelsep=.5cm,label=\bfseries\alph*)]
	\item $
	\begin{aligned}[t]
	i_1 &= -i_{25} \\
	&= \hlbox{-2\qw{\A}}
	\end{aligned} $
	\\[1cm]
	
	\item $
	\begin{aligned}[t]
	i_2 &= 2.5\qw{\A}+i_{100} \\
	&= \hlbox{3\qw{\A}}
	\end{aligned} $
	\\[1cm]
	
	\item $
	\begin{aligned}[t]
	i_3 &= i_{25}-10\qw{\A} \\
	&= \hlbox{-8\qw{\A}}
	\end{aligned} $
	\\[1cm]
	
	\item $
	\begin{aligned}[t]
	i_4 &= -i_{100} \\
	&= \hlbox{-0.5\qw{\A}}
	\end{aligned} $
	\\[1cm]
\end{enumerate}